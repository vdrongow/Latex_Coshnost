
\chapter{Einleitung}
\label{sec:Einleitung}

%in diesem Teil sehr wenig technisch und eher abstrakt
\chapter{Strategie}
\label{Strategie}
%welcher bot mit welchen Eigenschaften macht was 
%koop Strategie

\section{Bot 0}
\label{Bot0}

- Eigenschaften:
	- erh�ht den eigenen Farbanteil
	- ist am schnellsten
	- immer +1 (bis ins Unendliche)
- Strategie:
	- geht zu eigenen Farbanteil
	- alt. Strategie: evtl. Kreise fahren und dann bisschen zur Seite
	
\paragraph{Eigenschaften}

\section{Bot 1}
\label{Bot1}

- Eigenschaften:
	- setzt den eigenen Farbanteil hoch
	- ist am langsamsten
	- setzt immer wert auf =2
- Strategie:
	- sucht leere Farbfelder
	- wenn: egal welcher Wert das Feld hat, setzt auf wert 2, dann Ziel alle Felder aus�er die eigenen

\section{Bot 2}
\label{Bot2}

- Eigenschaften:
	- l�scht alle Farben inklusive der eigenen
	- kann sich �ber Gr�ben bewegen
- Strategie:
	- geht zu anderen Farben
	- verfolgt Bot 0 des besten Spielers
	- umgeht eigene Farben und Gr�ben

%a hier technisches Vorgehen
\chapter{Clustering}

\section{Datenstruktur}

%Daten vom Server

\section{Algorithmus}

\subsection{Parameter}

\subsection{Implementierung}

\section{Multithreading}

\chapter{Wege-Suche}
%was mache ich mit den Informationen

\section{Bewegung der Bots}

\section{Umgang mit Hindernissen}

% verschiedene bots bewegen sich unterschiedlich?

\chapter{Analyse}

\section{Zeit Messungen}