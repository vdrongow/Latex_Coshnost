
\chapter{Einleitung}
\label{sec:Einleitung}

%in diesem Teil sehr wenig technisch und eher abstrakt
\chapter{Strategie}
\label{Strategie}
%welcher bot mit welchen Eigenschaften macht was 
%koop Strategie
Dieses Kapitel behandelt die Strategie, die dem Verhalten der drei Bots zugrunde liegt. In diesem Teil wird die Idee wenig technisch und eher abstrakt erl�utert. \\
Jeder Bot hat bestimmte Eigenschaften, die in ein zuvor definiertes und teilweise aufeinander aufbauendes Verhalten resultieren. Im folgenden wird also jeder Bot mit seinen Eigenschaften und der Strategie beschrieben. 

\section{Bot 0}
\label{Bot0}
	
\paragraph{Eigenschaften}
Bot 0 erh�ht den eigenen Farbanteil eines Feldes immer um eins, wenn er dieses Feld passiert. Dabei ist das Inkrementieren nicht nach oben begrenzt. Er k�nnte den eigenen Farwert in einem Feld also theoretisch bis ins Unendliche erh�hen. \\
Gleichzeitig ist dieser Bot der schnellste von allen dreien. Diese Eigenschaften machen ihn sehr gut geeignet, um bereits eigene eingef�rte Felder zu verst�rken. 

\paragraph{Strategie}
Die Strategie ist auf die Eigenschaften des Bots angepasst. Dadurch, dass er sehr schnell ist, kann er in k�rzester Zeit gro�e Strecken auf der Spielkugel zur�cklegen. Wenn ein Feld jedoch schon von einer anderen Farbe eingef�rbt ist, kann dieser Bot nur bedingt etwas ausrichten. Daher bewegt er sich immer zu Bereichen, die der eigenen Farbe geh�ren. \\
Auf diesen Wert kann er den Wert erh�hen und somit ist der Farbanteil dort st�rker vertreten. Das bedeutet die Farbe flie�t st�rker auf umliegende Felder und sie bleibt auch l�nger auf dem Ursprungsfeld bestehen.

\section{Bot 1}
\label{Bot1}

- Eigenschaften:
	- setzt den eigenen Farbanteil hoch
	- ist am langsamsten
	- setzt immer wert auf =2
- Strategie:
	- sucht leere Farbfelder
	- wenn: egal welcher Wert das Feld hat, setzt auf wert 2, dann Ziel alle Felder aus�er die eigenen

\section{Bot 2}
\label{Bot2}

- Eigenschaften:
	- l�scht alle Farben inklusive der eigenen
	- kann sich �ber Gr�ben bewegen
- Strategie:
	- geht zu anderen Farben
	- verfolgt Bot 0 des besten Spielers
	- umgeht eigene Farben und Gr�ben

%a hier technisches Vorgehen
\chapter{Clustering}

\section{Datenstruktur}

%Daten vom Server

\section{Algorithmus}

\subsection{Parameter}

\subsection{Implementierung}

\section{Multithreading}

\chapter{Wege-Suche}
%was mache ich mit den Informationen

\section{Bewegung der Bots}

\section{Umgang mit Hindernissen}

% verschiedene bots bewegen sich unterschiedlich?

\chapter{Analyse}

\section{Zeit Messungen}