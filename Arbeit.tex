
%allgemeine Formatangaben
\documentclass[
 a4paper, 										% Papierformat
 12pt,												% Schriftgr��e
 ngerman, 										% f�r Umlaute, Silbentrennung etc.
 titlepage,										% es wird eine Titelseite verwendet
 bibliography=totoc,					% Literaturverzeichnis im Inhaltsverzeichnis auff�hren
 listof=totoc,								% Verzeichnisse im Inhaltsverzeichnis auff�hren
 oneside, 										% einseitiges Dokument
 captions=nooneline,					% einzeilige Gleitobjekttitel ohne Sonderbehandlung wie mehrzeilige Gleitobjekttitel behandeln
 numbers=noenddot,						% �berschriften-??Nummerierung ohne Punkt am Ende
 parskip=half									% zwischen Abs�tzen wird eine halbe Zeile eingef�gt
 ]{scrbook}


\newcommand{\titel}{CoShNoSt - Echtzeit Strategie}
\newcommand{\arbeit}{Projektbericht}
\newcommand{\modulname}{GT1 - AI for Games and Interactive Systems }
\newcommand{\hochschule}{Hochschule f�r Technik und Wirtschaft Berlin}
\newcommand{\fakultaet}{Fachbereich 4}
\newcommand{\autorA}{Matthias Hamborg}
\newcommand{\autorB}{Vivienne Drongowski}
\newcommand{\studiengang}{Masterstudiengang Internationale Medieninformatik}
\newcommand{\erstgutachter}{Prof. Dr. Lenz}
\newcommand{\ort}{Berlin}
\newcommand{\matrikelnr}{581491}
 			% einbinden von pers�nlichen Daten



% Anpassung an Landessprache
\usepackage[ngerman]{babel}
 
% Verwenden von Sonderzeichen und Silbentrennung
\usepackage[latin1]{inputenc}	
\usepackage[T1]{fontenc}			
\usepackage{textcomp} 																% Euro-Zeichen und andere
\usepackage[babel,german=quotes]{csquotes}						% Anf�hrungszeichen
\RequirePackage[ngerman=ngerman-x-latest]{hyphsubst} 	% erweiterte Silbentrennung

% Befehle aus AMSTeX f�r mathematische Symbole z.B. \boldsymbol \mathbb
\usepackage{amsmath,amsfonts}

% Zeilenabst�nde und Seitenr�nder 
\usepackage{setspace}
\usepackage{geometry}

% Einbinden von JPG-Grafiken
\usepackage{graphicx}

% zum Umflie�en von Bildern
% Verwendung unter http://de.wikibooks.org/wiki/LaTeX-Kompendium:_Baukastensystem#textumflossene_Bilder
\usepackage{floatflt}

% Verwendung von vordefinierten Farbnamen zur Colorierung
% Palette und Verwendung unter http://kitt.cl.uzh.ch/kitt/CLinZ.CH/src/Kurse/archiv/LaTeX-Kurs-Farben.pdf
\usepackage[usenames,dvipsnames]{color} 

% Tabellen
\usepackage{array}
\usepackage{longtable}

% einfache Grafiken im Code
% Einf�hrung unter http://www.math.uni-rostock.de/~dittmer/bsp/pstricks-bsp.pdf
\usepackage{pstricks}

% Quellcodeansichten
\usepackage{verbatim}
\usepackage{moreverb} 											% f�r erweiterte Optionen der verbatim Umgebung
% Befehle und Beispiele unter http://www.ctex.org/documents/packages/verbatim/moreverb.pdf
\usepackage{listings} 											% f�r angepasste Quellcodeansichten siehe
% Kurzeinf�hrung unter http://blog.robert-kummer.de/2006/04/latex-quellcode-listing.html

% Glossar und Abbildungsverzeichnis
\usepackage[
nonumberlist, %keine Seitenzahlen anzeigen
acronym,      %ein Abk�rzungsverzeichnis erstellen
toc          %Eintr�ge im Inhaltsverzeichnis
]      %im Inhaltsverzeichnis auf section-Ebene erscheinen
{glossaries}

% verlinktes und Farblich angepasstes Inhaltsverzeichnis
\usepackage[pdftex,
colorlinks=true,
linkcolor=InterneLinkfarbe,
urlcolor=ExterneLinkfarbe]{hyperref}
\usepackage[all]{hypcap}

% URL verlinken, lange URLs umbrechen
\usepackage{url}

% sorgt daf�r, dass Leerzeichen hinter parameterlosen Makros nicht als Makroendezeichen interpretiert werden
\usepackage{xspace}

% Beschriftungen f�r Abbildungen und Tabellen
\usepackage{caption}

% Entwicklerwarnmeldungen entfernen
\usepackage{scrhack}					% einbinden der verwendeten Latex-Pakete


\onehalfspacing 							% 1,5facher Zeilenabstand

\definecolor{InterneLinkfarbe}{rgb}{0.1,0.1,0.3} 	% Farbliche Absetzung von externen Links
\definecolor{ExterneLinkfarbe}{rgb}{0.1,0.1,0.7}	% Farbliche Absetzung von internen Links

% Einstellungen f�r Fu�noten:
\captionsetup{font=footnotesize,labelfont=sc,singlelinecheck=true,margin={5mm,5mm}}

% Stil der Quellenangabe
\bibliographystyle{alphadin}

%Ausschluss von Schusterjungen
\clubpenalty = 10000
%Ausschluss von Hurenkindern
\widowpenalty = 10000

% Befehle, die Umlaute ausgeben, f�hren zu Fehlern, wenn sie hyperref als Optionen �bergeben werden
\hypersetup{
%    pdftitle={\titel \untertitel},
%    pdfauthor={\autor},
%    pdfcreator={\autor},
%    pdfsubject={\titel \untertitel},
%    pdfkeywords={\titel \untertitel},
}

% Beispiel f�r eine Listings-Codeumbebungen
% Bei mehreren Definitionen empfielt sich das auslagern in eine externe Datei
\lstloadlanguages{Java,HTML}
\lstset{
	frame=tb,
	framesep=5pt,
	basicstyle=\footnotesize\ttfamily,
	showstringspaces=false,
	keywordstyle=\ttfamily\bfseries\color{CadetBlue},
	identifierstyle=\ttfamily,
	stringstyle=\ttfamily\color{OliveGreen},
	commentstyle=\color{GrayBlue},
	rulecolor=\color{Gray},
	xleftmargin=5pt,
	xrightmargin=5pt,
	aboveskip=\bigskipamount,
	belowskip=\bigskipamount
} 

%Den Punkt am Ende jeder Beschreibung deaktivieren
\renewcommand*{\glspostdescription}{}

% Empfehlung: Abkuerzungsverzeichnis und Glossar sind in Graduierunsarbeiten
% nicht zwingend notwendig

% %Glossar-Befehle anschalten
% \makeglossaries
% \glsenablehyper
% \input{Header/Abkuerzungen}
% \input{Header/Glossar}


\areaset[0.5cm]{16cm}{24cm}


\begin{document}

%\include{Deckblatt/titel}						% Deckblatt der vorliegenden Arbeit
% Die Daten werden in Header/Metadaten.tex eingstellt
\begin{titlepage}
\begin{Large}
\begin{center}

\includegraphics[width=0.8\textwidth]{Abbildungen/HTW_Berlin_Logo_sw_quer.jpg}
% \textbf{\hochschule\ \ort}\\[5pt]
% \fachbereich\\
% \studiengang\\
% \vskip 1cm
% \arbeit\\
% zur Erlangung der akademischen Grades
\\[50pt]

\arbeit\\[-1mm]
\modulname\\[3mm]
im \studiengang\\[-1mm]
des \fakultaet%\\[-1mm]
%der \hochschule  %% kann entfallen
\\
\vfill
{\LARGE\bfseries \titel \par}
\vfill

vorgelegt von\\
 \autorB
\\Matrikelnummer: \matrikelnr  
\\[8pt]
programmiert von\\
\autorA  \ und \autorB
\\[15pt]
\ort, den \today

\end{center}
\vfill
\begin{tabular}{ll}
Pr�fer: & \erstgutachter\\
\end{tabular}
\end{Large}
\end{titlepage}

\frontmatter												% Seitenzählerstart vor dem Text

%
\chapter*{Abstrakt}
\label{sec:Abstrakt}
						% Abstrakt

%% Empfehlung: Auf Danksagungen + Vorwort in wissenschaftlichen Arbeiten eher verzichten -- ggf. die entsprechenden Dateien im Ordner Inhalt anlegen und hier auskommentieren (PKW)
%% \include{Inhalt/Danksagung}					% Danksagung
%% \include{Inhalt/Vorwort}						% Vorwort

\tableofcontents										% Inhaltsverzeichnis

\mainmatter													% Seitenzählerstart Haupttext


\chapter{Einleitung und Intension}
\label{sec:EinleitungUndIntension}
Was m�chte ich sagen?

\chapter{Erste �berschrift}
\label{sec:ErsteUeberschrift}
Hier steht der erste Text.

\section{Erstes Unterkapitel}
\label{sec:ErstesUnterkapitel}
Und hier wird es schon spezifischer.


\section{Zweites Unterkapitel}
\label{sec:ZweitesUnterkapitel}

In der Regel reichen zwei Gliederungsebenen f�r Graduierungsarbeiten aus.					% hier steht der eigentliche Text der Arbeit

\pagenumbering{Roman}
\appendix	
												
\bibliography{Literatur/Literatur} 	% Literaturverzeichnis

% Empfehlung: Verzeichnisse an das Ende der Arbeit (PKW)
% Ein Glossar ist eher untypisch fuer Graduierungsarbeiten
% Abkuerzungsverzeichnis an das Ende der Arbeit; Abkuerzungen und Begriffe muessen in jedem
% Fall im Haupttext der Arbeit beim ersten Vorkommen definiert werden

%% \printglossary[title=Glossar] 			% Glossar Einträge in Header/Glossar.tex vornehmen
%%\printglossary[type=\acronymtype,title=Abk\"urzungsverzeichnis]	% Abkürzungsverzeichnis Einträge in Header/Abkuerzungen vornehmen

\listoffigures											% Abbildungsverzeichnis
\listoftables												% Tabellenverzeichnis


\pagestyle{empty}

\chapter*{Selbst�ndigkeitserkl�rung}
\label{sec:Selbst�ndigkeitserkl�rung}

Ich erkl�re hiermit, dass ich die vorliegende Graduierungsarbeit ohne Hilfe Dritter und ohne Benutzung anderer als der angegebenen Quellen und Hilfsmittel verfasst habe. Alle den benutzten Quellen w�rtlich oder sinngem�� entnommene Stellen sind als solche einzeln kenntlich gemacht.

Diese Arbeit ist bislang keiner anderen Pr�fungsbeh�rde vorgelegt und auch nicht ver�ffentlicht worden.

Ich bin mir bewusst, dass eine falsche Erkl�rung rechtliche Folgen haben wird.\\[1.5cm]


\begin{minipage}{0.7\textwidth}
\ort, \today \hfill  Unterschrift
\end{minipage}





\end{document}